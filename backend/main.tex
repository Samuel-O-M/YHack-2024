\documentclass{beamer}
\usetheme{Madrid}
\usepackage{amsmath}
\usepackage{graphicx}
\title{Mathematics Concepts}
\subtitle{}
\author{John Doe}
\institute{}
\date{2021-01-01}
\begin{document}
\begin{frame}
\titlepage
\end{frame}
\begin{frame}{Table of Contents}
\tableofcontents
\end{frame}
\section{Introduction}
\begin{frame}
\centering
\Huge{Introduction}
\end{frame}
\begin{frame}{Introduction}
\begin{itemize}
    \item \textbf{Quadratic Function:} 
    \begin{align*}
    f(x) = x^2
    \end{align*}
    \begin{itemize}
        \item Represents a parabola opening upwards.
        \item Commonly used in optimization problems.
    \end{itemize}
    
    \item \textbf{Linear Function:}
    \begin{align*}
    g(x) = 2x + 5
    \end{align*}
    \begin{itemize}
        \item Represents a straight line with slope 2.
        \item Indicates a constant rate of change.
    \end{itemize}

    \item \textbf{Cubic Function:}
    \begin{align*}
    p(x) = x^3 - 4x + 1
    \end{align*}
    \begin{itemize}
        \item Represents a curve that can have one or two turning points.
        \item Used in various physical and economic models.
    \end{itemize}
\end{itemize}\end{frame}
\begin{frame}{Introduction}
\end{frame}
\section{Formulas}
\begin{frame}
\centering
\Huge{Formulas}
\end{frame}
\begin{frame}{Formulas}
\begin{align*}f(x) = x^2\end{align*}\end{frame}
\begin{frame}{Formulas}
\begin{align*}g(x) = 2x + 5\end{align*}\begin{align*}p(x) = x^3 - 4x + 1\end{align*}\begin{align*}p(x) = x^3 - 4x + 1\end{align*}\end{frame}
\section{Images}
\begin{frame}
\centering
\Huge{Images}
\end{frame}
\begin{frame}{Images}
\begin{figure}[h]\centering\includegraphics[width=0.8\textwidth]{image1.png}\end{figure}\begin{figure}[h]\centering\includegraphics[width=0.8\textwidth]{icon.png}\end{figure}\end{frame}
\section{Conclusion}
\begin{frame}
\centering
\Huge{Conclusion}
\end{frame}
\begin{frame}
\centering
\Huge{Thank You!}
\end{frame}
\end{document}
